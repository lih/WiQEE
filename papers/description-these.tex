\documentclass[]{easychair}
\usepackage{lmodern}
\usepackage{amssymb,amsmath}
\usepackage{ifxetex,ifluatex}
\usepackage{fixltx2e} % provides \textsubscript
\ifnum 0\ifxetex 1\fi\ifluatex 1\fi=0 % if pdftex
  \usepackage[T1]{fontenc}
  \usepackage[utf8]{inputenc}
\else % if luatex or xelatex
  \ifxetex
    \usepackage{mathspec}
    \usepackage{xltxtra,xunicode}
  \else
    \usepackage{fontspec}
  \fi
  \defaultfontfeatures{Mapping=tex-text,Scale=MatchLowercase}
  \newcommand{\euro}{€}
\fi
% use upquote if available, for straight quotes in verbatim environments
\IfFileExists{upquote.sty}{\usepackage{upquote}}{}
% use microtype if available
\IfFileExists{microtype.sty}{%
\usepackage{microtype}
% \usepackage[usenames,dvipsnames]{color}
\UseMicrotypeSet[protrusion]{basicmath} % disable protrusion for tt fonts
}{}
\ifxetex
  \usepackage[setpagesize=false, % page size defined by xetex
              unicode=false, % unicode breaks when used with xetex
              xetex]{hyperref}
\else
  \usepackage[]{hyperref}
\fi
\hypersetup{breaklinks=true,
            bookmarks=true,
            pdfauthor={Marc Coiffier},
            pdftitle={Thèse : Définition et implémentation de schémas calculatoires pour la preuve de programmes},
            colorlinks=true,
            citecolor=blue,
            urlcolor=blue,
            linkcolor=magenta,
            pdfborder={0 0 0}}
\urlstyle{same}  % don't use monospace font for urls
\setlength{\parindent}{0pt}
\setlength{\parskip}{6pt plus 2pt minus 1pt}
\setlength{\emergencystretch}{3em}  % prevent overfull lines
\providecommand{\tightlist}{%
  \setlength{\itemsep}{0pt}\setlength{\parskip}{0pt}}
\setcounter{secnumdepth}{0}

\title{Thèse : Définition et implémentation de schémas calculatoires pour la
preuve de programmes}
\author{Marc Coiffier}
\date{}
\DeclareUnicodeCharacter{03BB}{\lambda}
\DeclareUnicodeCharacter{2200}{\forall}

% Redefines (sub)paragraphs to behave more like sections
\ifx\paragraph\undefined\else
\let\oldparagraph\paragraph
\renewcommand{\paragraph}[1]{\oldparagraph{#1}\mbox{}}
\fi
\ifx\subparagraph\undefined\else
\let\oldsubparagraph\subparagraph
\renewcommand{\subparagraph}[1]{\oldsubparagraph{#1}\mbox{}}
\fi

\begin{document}
\maketitle

{
\hypersetup{linkcolor=black}
\setcounter{tocdepth}{3}
\tableofcontents
\newpage
}

\emph{À faire} :

\begin{itemize}
\tightlist
\item
  renseigner machine CAM
\item
  développer le COq -\textgreater{} Mu
\item
  (XAH pour taper les caractères unicode)
\item
  Justifier LAP à partir de Ltac / SsReflect

  \begin{itemize}
  \tightlist
  \item
    Portée dynamique pourquoi ? En 1 et deux mots
  \end{itemize}
\item
  Réordonner : motivations concrètes d'abord
\end{itemize}

Ma thèse se faisant dans le cadre du project VOCAL (Verified OCAml
Library, pour les intimes), j'ai tout d'abord découvert l'univers de la
preuve formelle assistée par ordinateur, en me familiarisant avec le
code dudit projet. Cette familiarisation a été principalement dirigée
par la volonté d'y intégrer une preuve de la correction de l'algorithme
MergeSort implémenté dans la librairie standard OCaml.

Une fois familiarisé avec les bases d'un assistant de preuve (Coq), et
de son modèle de calcul, j'ai exploré différentes modélisations de
théories mathématiques établies, afin de me faire une idée des limites
que l'on pouvait rencontrer, et peut-être de trouver une contribution
intéressante à apporter.

\subsection{Expériences avec Coq et
Ltac}\label{expuxe9riences-avec-coq-et-ltac}

L'environnement de preuve interactif de Coq repose sur deux langages :

\begin{itemize}
\item
  le calcul des construction inductives (alias CIC), qui par
  l'isomorphisme de Curry-Howard permet de justifier la véracité de
  propriétés mathématique en fournissant des programmes qui
  ``implémentent'' ces propriétés.

  L'existence d'un terme de type \(P\) suffit, en CIC, à prouver la
  propriété \(P\) (de façon plus générale, on peut supposer que la
  propriété et sa preuve partagent un contexte \(\Gamma\), qui peut
  donner accès à certaines valeurs ou propriétés supplémentaires sous
  forme d'hypothèses)
\item
  un langage de tactiques, abrégé Ltac, qui permet de construire des
  termes en CIC en suivant les propriétés que l'on cherche à prouver,
  qui deviennent des \emph{objectifs} de preuve (en anglais, celà donne
  du \emph{goal-oriented programming}).

  En Ltac, on se préoccupe de raffiner des objectifs de preuve jusqu'à
  la trivialité, c'est-à-dire jusqu'à que les propriétés que l'on
  cherche à prouver soient fournies dans le contexte.
\end{itemize}

D'un point de vue linguistique, Ltac réifie les objets du CIC, ce qui
permet une manipulation de premier ordre des contextes et des preuves
lors du développement de ces dernières. Ltac permet également la
manipulation d'objets qui ne sont pas directement liés à la preuve, tels
les entiers naturels et les noms d'hypothèses (dans la tactique
\texttt{fix\ \textless{}n\textgreater{}\ \textless{}hyp\textgreater{}},
par exemple), ainsi que les tactiques elles-même, qui peuvent êtres
stockées dans des ``tactic objects'', et utilisées de manière similaire
aux tactiques natives.

D'avril à juillet 2018, à la fin de ma première année, j'ai pu utiliser
cette réification pour implémenter une tactique de ``petites
inversions'', qui permet l'inversion de certaines hypothèses de manière
plus concise que la tactique d'inversion générale, sans introduire de
preuves d'égalité comme cette dernière, et parfois de façon à éviter une
dépendance sur l'axiome K lors d'inductions fortement dépendantes.

Lors de l'écriture de cette tactique, j'ai pu identifier certaines
limites à la capacité de Ltac à inspecter des termes arbitraires, tout
particulièrement des points fixes. En effet, les points fixes de Coq
font l'objet de contraintes structurelles qui se prêtent mal à la
déconstruction, de part la présence de paramètres privilégiés (dits
\emph{paramètres structurels}) qui se doivent de donner lieu à une
preuve de décroissance structurelle lors de l'instanciation du point
fixe considéré sur un argument inductif concret.

De plus, Ltac étant un langage principalement dédié à l'exploration de
plusieurs alternatives lors de la construction de preuves, il ne s'est
pas doté de primitives simples pour la définition de structures de
données hors du champ du CIC, ce qui complique considérablement son
utilisation lors de l'écriture de scripts complexes. Les seules
abstractions dont il dispose pour ce faire sont les ``fonctions
tactiques'', des sortes de lambdas non-typées, qui permettent donc de
retrouver cette expressivité manquante par encodage de Church et autres
machinations peu recommandables.

Ces limites (et d'autres plus techniques), à la fois dans le CIC de Coq
et dans son langage de tactique, m'ont conduit à une réflexion sur la
conception d'un environnement de preuve similaire, mais capable d'offrir
à la fois un calcul capable de toutes les introspections, et un langage
de manipulation de termes dans lequel l'exploration de nouvelles
tactiques pourrait se faire plus naturellement.

\subsection{Limites du CIC simple}\label{limites-du-cic-simple}

Lors de cette réflexion, j'ai voulu modéliser des \(\omega\)-catégories
(si je ne m'abuse sur les nomenclatures), qui peuvent être définies
intuitivement à l'aide des structures suivantes :

Soient \(O\) un type d'objets, et pour toute paire d'objets \(x\) et
\(y\), un type \(M x y\) des 0-morphismes de \(x\) vers \(y\). On
aimerait définir les familles de types inductifs \(V_{n} : Type\) et
\(M_{n} : V_{n} \rightarrow V_{n} \rightarrow Type\) (resp. des n-objets
et n-morphismes de notre \(\omega\)-catégorie), dotés des constructeurs
suivants :

\begin{align*}
v_0 &: O \rightarrow V_{0} \\
v_S &: \forall n (x y : V_{n}), M_{n} x y \rightarrow V_{S n} \\
m_0 &: \forall (x y : O), M x y \rightarrow M_{0} (v_0 x) (v_0 y) \\
m_S &: \forall n (x y z t : V_{n}) (f : M_{n} x y) (g : M_{n} z t), M_{n} x z \rightarrow M_{n} y t \rightarrow M_{S n} (v_S\,n\,x\,y\,f) (v_S\,n\,z\,t\,g) 
\end{align*}

Coq (et d'autres assistants basés sur le CIC) ne permet pas la
définition de familles mutuellement inductives (comme les \(V_n\) et
\(M_n\) définis ci-dessus) si l'une des familles doit servir d'index à
l'autre. Cette limite est justifiée par l'apparente impossibilité de
faire référence à un constructeur dans le type d'un autre constructeur
si les deux appartiennent à la même famille.

Il est curieux de trouver de telles limites sur les familles inductives
mutuellement récursives, puisqu'il est plutôt simple -- dans un contexte
de types dépendants -- d'en définir un encodage de Church
(\(O_{n} \equiv \forall (O:\mathbb{N} \rightarrow Type) (M:\forall i, O i \rightarrow O i \rightarrow Type) (o0 : o \rightarrow O 0) (oS : ...) ..., O n\),
par exemple).

Bien entendu, l'encodage de Church n'est pas une panacée, car il ne
donne pas accès à un grand nombre des propriétés qui nous intéressent
dans l'étude de valeurs inductives (la discrimination des constructeurs,
pour n'en citer qu'une). Le CIC, et ses dérivés, ne disposent pas à ce
jour de façon satisfaisante de définir ces familles.

\subsection{\texorpdfstring{Solution en travail : une autre extension du
CoC, capable de gérer ``plus''
d'inductifs}{Solution en travail : une autre extension du CoC, capable de gérer plus d'inductifs}}\label{solution-en-travail-une-autre-extension-du-coc-capable-de-guxe9rer-plus-dinductifs}

Dans l'optique de combler cette lacune, je cherche à donner aux CoC la
capacité de prouver des principes d'induction directement sur des
encodages de Church, plutôt que sur des types ``définis inductifs''.

Pour ce faire, je me fie à une observation qui semble se vérifier en
pratique : les principes d'induction ont une forme qui reflètent celle
de l'encodage de Church de l'inductif sur lequel ils travaillent. Par
exemple, le principe d'induction des booléens (\texttt{Boolean\_rect} en
Coq) a le type suivant :

\[
\forall (P:Boolean \rightarrow Type), P true \rightarrow P false \rightarrow \forall (b:Boolean), P b
\]

Le paramètre \(b\) ne dépendant d'aucun des paramètres précédents, on
peut le décaler à gauche du \(\forall\) principal, et l'abstraire dans
le contexte :

\[
\forall (P:Boolean \rightarrow Type), P true \rightarrow P false \rightarrow P b
\]

À présent, si on regarde l'encodage de Church des mêmes booléens, on a
le type suivant :

\[
\forall (P:Type), P \rightarrow P \rightarrow P
\]

Autrement dit, la même structure, moins un paramètres booléen ! Bien
sûr, les choses sont plus compliquées dans le cas de types récursifs
(tels les naturels), car il s'agit de mêler discrimination et récursion
dans la sémantique opérationnelle des principes d'induction.

Étant donné un encodage de Church, on peut faire le travail inverse, et
générer automatiquement le type du principe d'induction canonique de cet
encodage. Dans ma thèse, j'introduis un nouvel opérateur, noté
\(\mu(x)\), dont c'est le rôle : étant donné une valeur inductive \(x\),
produire un terme dont la structure ``reflète'' celle de \(x\), en y
ajoutant les paramètres nécessaires à la preuve du principe d'induction
spécialisé sur \(x\).

Un interpréteur de ce nouveau modèle de calcul -- CoC + \(\mu\), appelé
le Calcul des Constructions Prismatiques pour rappeler la métaphore des
``reflets with extra steps'' -- est rendu disponible sur une plateforme
Web interactive, ainsi que sous une forme plus classique, à l'adresse
\url{https://wiqee.curly-lang.org} .

\end{document}
